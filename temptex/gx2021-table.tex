\documentclass[a4paper,UTF8,zihao=-4]{ctexart}%zihao=-4
\usepackage{amsfonts,amsmath,amssymb,amsbsy,amsthm}
\usepackage{mathrsfs,mathtools}
\usepackage{bm}
\usepackage{braket}
\usepackage{advdate}% "\AdvYear{-2}\the\year "在真实年减去2年
%\usepackage{graphicx}
\usepackage{tabularx}%换行表格
%\usepackage{newtxtext,newtxmath}%正文与数字均罗马字体,特殊符号仍然好看
\usepackage{newtxtext}%正文罗马字体,不影响数学环境字体
\setCJKmainfont[BoldFont=SimHei]{SimSun}
\usepackage[top=2.0cm,bottom=2.54cm,left=2.4cm,right=2.4cm]{geometry}
\newcommand{\alert}[1]{\textcolor{blue}{#1}}%定义\alert{}变蓝不加粗
%%%%%%
\makeatletter
\newcommand\swustunderline[2][4cm]{\hskip1pt\underline{\hb@xt@#1{\hss#2\hss}}\hskip3pt}%定义下划线
\makeatother
%%%%
\newcommand{\thecourse}{量子力学}%名称%LX190620
\newcommand{\thephase}{课带}%构思、设计、实施
\newcommand{\mythesis}{•}%本组项目名称
%\newcommand{\mythesiss}{}%长题目分成2段
\newcommand{\gnumber}{第•组}%组号,用中文数字
\newcommand{\charger}{•}%组长名
\newcommand{\members}{•}%组员,以~分开
\newcommand{\myschool}{理学院}%
\newcommand{\mymajor}{光电信息科学与工程系}%
\newcommand{\supervisor}{黎雷}%指导教师
\newcommand{\mytitle}{\mythesis}%页眉内容%\mythesiss
\newcommand{\thisgrade}{\AdvYear{-1}\the\year}
\newcommand{\numbtt}{51\thisgrade}
\newcommand{\thecredit}{\numbtt•}%组长学号
%%%
%%tex.stackexchange.com/questions/458168/how-to-get-the-last-two-digits-of-next-year
\newcommand{\lasttwoyear}[1]{% #1 is the offset
  \expandafter\getlasttwo\number\numexpr\year+(#1)\relax\relax
}
\def\getlasttwo#1#2#3#4\relax{#3#4}
%%自定义的\lasttwoyear{}命令,{}里面可填参数-2,-1,+1等,只提取年份后两位
%%
\newcommand{\classgx}{光信\lasttwoyear{-1}}%理学院光信XX级简写
%%%
%%%%
\begin{document}
%\title{课带项目成绩表格}
%\date{}
%\maketitle{}
%\tableofcontents

\begin{center}
  {\LARGE 《\thecourse》\thephase 项目信息表}%可以不用\bf加粗
\end{center}

\begin{table}[h!]
  \centering
  \begin{tabular}[1\textwidth]{lclc}%
{\bf 授课时间:}
&\swustunderline[135pt]{\ifthenelse{\the\month < 9}{\ifthenelse{\the\month < 3}{{\AdvYear{-1}\the\year}-\the\year 学年第1学期}{{\AdvYear{-1}\the\year}-\the\year 学年第2学期}}{\the\year-{\AdvYear{+1}\the\year} 学年第1学期}}
&{\bf 教学单位:}
&\swustunderline[90pt]{\myschool}\\[3mm]
{\bf 项目名称:}&\multicolumn{3}{c}{\swustunderline[310pt]{\bf \mythesis }}\\[3mm]%\mythesiss
%&{\bf 教师:}&\swustunderline[46pt]{\supervisor}\\
{\bf 学生班级:}
&\swustunderline[135pt]{光信\thisgrade}
&{\bf 指导教师:}
&\swustunderline[90pt]{\supervisor}\\[3mm]
{\bf 负责人:}
&\swustunderline[135pt]{\charger}
&{\bf 学\qquad 号:}
&\swustunderline[90pt]{\thecredit}\\[3mm]
{\bf 团队成员:}&\multicolumn{3}{c}{\swustunderline[310pt]{\members}}\\
  \end{tabular}
\end{table}

{\fontsize{12pt}{14pt}\selectfont%
%\begin{table}[h!]
  \centering
  \begin{tabular}[1\textwidth]{|m{1em}<{\centering}|m{4em}<{\centering}|m{5em}<{\centering}|m{7em}<{\centering}|m{3em}<{\centering}|m{3em}<{\centering}|m{3em}<{\centering}|m{5em}<{\centering}|}

\multicolumn{8}{l}{\textbf{\gnumber\quad 组员成绩表}($ • \times0.6 + • \times 0.4 = • ,\quad • \times • $)}\\  % 列宽用% m可实现上下居中,而用p{1.3cm}则不能
  \hline                              
  
序&姓名&班级&学号&成绩1&成绩2&成绩3&总成绩\\%%成绩1 满分30, 成绩2 满分40, 成绩3 满分30
    \hline
 1 & 赵同学 & \classgx01 & \numbtt1111 & • & • & • & • \\
    \hline
 2 & 钱同学 & \classgx01 & \numbtt2222 & • & • & • & • \\
    \hline
 3 & 孙同学 & \classgx02 & \thecredit & • & • & • & • \\
    \hline
 4 & 李同学 & \classgx02 & \numbtt4444 & • & • & • & • \\
    \hline
 5 & 周同学 & \classgx01 & \numbtt5555 & • & • & • & • \\
    \hline
 6 & 吴同学 & \classgx02 & \numbtt6666 & • & • & • & • \\
    \hline
 7 & 郑同学 & \classgx01 & \numbtt7777 & • & • & • & • \\
    \hline
 8 & 王同学 & \classgx02 & \numbtt8888 & • & • & • & • \\
    \hline
  \multicolumn{8}{l}{}\\ 
      \hline
       \multicolumn{8}{|l|}{\bf 教师评语:}\\        
     \multicolumn{8}{|l|}{\rule{0mm}{90mm}} \\
     \multicolumn{8}{|l|}{\hfill{成绩评定:\rule{30mm}{0mm}}}\\ 
     \multicolumn{8}{|l|}{\hfill{指导教师(签名):\rule{30mm}{0mm}}}\\ 
     \multicolumn{8}{|l|}{\hfill{ \the\year 年 \the\month 月 \AdvanceDate[-3]\the\day 日\rule{15mm}{0mm}}}\\
     \hline
  \end{tabular}
%\end{table}
}

\thispagestyle{empty}
\clearpage
%%
\end{document}