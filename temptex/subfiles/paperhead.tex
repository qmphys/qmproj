\usepackage{amsfonts,amssymb,amsbsy}%
\usepackage[many]{tcolorbox}%可以换行的colorbox,many选项含自动加载amsmath
\tcbuselibrary{vignette}%tcolorbox的many选项没有加载这个库
\usepackage{mathrsfs,mathtools}
\usepackage{bm}%数学字体加粗
\usepackage{braket}%狄拉克括号
\usepackage{tabularx}%增强表格
\usepackage{multirow}%跨行表格
\usepackage{longtable}%可跨页长表格
\usepackage[version=4]{mhchem}%输入元素符号,文本、数学环境皆可
\usepackage{graphicx}
\usepackage{enumitem}%个性化enumerate
\usepackage{xcolor}%可以使用颜色名xcolor
\usepackage{advdate}% "\AdvYear{-2}\the\year"在真实年减去2年
\usepackage{listings}%编程代码显示
\usepackage[left=2.4cm,right=2.4cm,top=2.5cm,bottom=2.5cm]{geometry}%定义页边距
\usepackage{extarrows}%可输入增长的数学箭头
%%%
\usepackage{fancyhdr}%页面设置定制
\pagestyle{fancy}
\fancyhf{}%普通样式
\fancyhead[CO]{西南科技大学课带项目报告}
\fancyhead[CE]{\mytitle}
\fancyfoot[C]{\thepage}
\renewcommand{\headrulewidth}{.5pt}%页眉的横线宽度
%双线页眉的设置 
\makeatletter %双线页眉
\def\headrule{{\if@fancyplain\let\headrulewidth\plainheadrulewidth\fi% 
		\hrule\@height 0.6pt \@width\headwidth\vskip1pt%上面线为1pt粗   
		\hrule\@height 1.0pt\@width\headwidth  %下面0.5pt粗         
		\vskip-2\headrulewidth\vskip-1.5pt}      % 两条线的距离1pt
	\vspace{6mm}}     %双线与下面正文之间的垂直间距           
\makeatother
%%%
\usepackage[numbered]{bookmark}%自定义书签,例如把“目录”列入书签
%\usepackage{newtxtext,newtxmath}%正文与数字均罗马字体,特殊符号仍然好看
\usepackage{newtxtext}%正文罗马字体,不影响数学环境字体
\setCJKmainfont[BoldFont=SimHei,ItalicFont=KaiTi]{SimSun}
%%%
\usepackage{hyperref}%文本超链接
\hypersetup{
  colorlinks = true,   % Colours links instead of ugly boxes
  urlcolor   = blue,   % Colour for external hyperlinks
  linkcolor  = blue, % Colour of internal links
  citecolor  = red   % Colour of citations 注意最后一行没有,号
}
%%%
\graphicspath{{grafig/}}%%须配合上面的graphicx宏包使用
%%%定义下划线
\makeatletter
\newcommand\swustunderline[2][4cm]{\hskip1pt\underline{\hb@xt@#1{\hss#2\hss}}\hskip3pt}
\makeatother
%%%
\newcommand{\alert}[1]{\textcolor{blue}{#1}}%定义\alert{}变蓝不加粗
%%%%
%%%%
\newcommand{\thecourse}{量子力学}%名称%LX190620
\newcommand{\thephase}{课带}%构思、设计、实施
\newcommand{\mythesis}{•}%本组项目名称
%\newcommand{\mythesiss}{}%长题目分成2段
\newcommand{\gnumber}{第•组}%组号,用中文数字
\newcommand{\charger}{•}%组长名
\newcommand{\members}{•}%组员,以~分开
\newcommand{\myschool}{理学院}%
\newcommand{\mymajor}{光电信息科学与工程系}%
\newcommand{\supervisor}{黎雷}%指导教师
\newcommand{\mytitle}{\mythesis}%页眉内容%\mythesiss
\newcommand{\thisgrade}{\AdvYear{-1}\the\year}
\newcommand{\numbtt}{51\thisgrade}
\newcommand{\thecredit}{\numbtt•}%组长学号
%%%
%%tex.stackexchange.com/questions/458168/how-to-get-the-last-two-digits-of-next-year
\newcommand{\lasttwoyear}[1]{% #1 is the offset
  \expandafter\getlasttwo\number\numexpr\year+(#1)\relax\relax
}
\def\getlasttwo#1#2#3#4\relax{#3#4}
%%自定义的\lasttwoyear{}命令,{}里面可填参数-2,-1,+1等,只提取年份后两位
%%
\newcommand{\classgx}{光信\lasttwoyear{-1}}%理学院光信XX级简写
%%%
%%%%
%%%%